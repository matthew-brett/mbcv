% The skeleton for producing my publication list in sections Uses biblatex and
% biber to customized the bibliography output. Sections defined by a
% combination of keyword= and custom filtering. See
% http://www.ctan.org/pkg/biblatex

\documentclass{cv}
% Document margins
\usepackage[left=0.75in,top=0.6in,right=0.75in,bottom=0.6in]{geometry}
\usepackage[utf8]{inputenc}
\usepackage[T1]{fontenc}
\usepackage{mathpazo}
\usepackage{hyperref}
\usepackage{url}
\usepackage{footmisc}

\pagestyle{plain}
\pagenumbering{arabic}

\usepackage[defernumbers=true, bibstyle=authoryear, backend=biber, maxbibnames=10,
    url=false, dashed=false, sorting=ydnt]{biblatex}

\usepackage{hyperref}
\hypersetup{ % turn off the boxes round URLs
    colorlinks,
    citecolor=black,
    filecolor=black,
    linkcolor=black,
    urlcolor=black
}

\addbibresource{matthew_brett.bib}
\addbibresource{cv.bib}

\AtEveryBibitem{\clearlist{language}} % clears language

% This stuff below was the result of trial and error to get the bibliography
% output to a) preserve the subsequent paragraph indentation as indicated by
% the adjustwidth environments, and b) To have the correct indentation for
% each entry in the bibliography.  The source from which I was working was
% http://tex.stackexchange.com/questions/46298/printing-bibliography-with-biblatex-in-tufte-handout-fullwidth-environment
\defbibenvironment{bibliography}
  {\list{}{%
          \leftmargin\bibhang
  }}
  {\endlist}
    {\item}

\AtEveryBibitem{\hskip-\bibhang}

% Select imaging and statistics entries in bib database
\defbibfilter{imagingOrStatistics}{%
    keyword=imaging or keyword=statistics
}

\newcommand{\PlaceDate}[2]{{\bf #1} \hfill {\em #2} \\}
\newcommand{\PlaceDateNote}[3]{{\bf #1} \hfill {\em #2} \\#3}
\newcommand{\UCB}{University of California, Berkeley}
\newcommand{\CBU}{MRC cognition and brain sciences unit, Cambridge}
\newcommand{\Pkg}[1]{{\tt #1}}

\begin{document}

\nocite{*}

{\huge \bf Matthew Brett}

Helen Wills neuroscience institute \\
210 Barker Hall \\
University of California \\
Berkeley CA 94720 \\
matthew.brett@gmail.com

\begin{cvSection}{Education}

\PlaceDate{Royal London hospital}{1987--1990 }
Bachelor of medicine and surgery

\PlaceDateNote{Cambridge university}{1984--1987 }{
Open entrance scholarship \\
BA 2.{\em i}; Experimental psychology}

\end{cvSection}

\begin{cvSection}{Work history}

\PlaceDateNote{\UCB}{2008--present }{
Associate researcher at the brain imaging center}

\PlaceDateNote{\CBU}{2005--2008}{
Senior investigator scientist}

\PlaceDateNote{\UCB}{2003--2005 }{
Associate specialist in psychology; advised by Rich Ivry}

\PlaceDateNote{\CBU}{1999--2003 }{
Research associate in psychology; advised by John Duncan}

\PlaceDateNote{
MRC cyclotron unit, Hammersmith hospital / Physiology laboratory, Oxford}
{1996--1999}
{Research registrar in neurology; advised by David Brooks (London) and John
Stein (Oxford)}

\PlaceDateNote{Radcliffe infirmary, Oxford}
{1995--1999}
{Neurology registrar}

\PlaceDateNote{National hospital for neurology, London}{1994--1995 }{
Neurology senior house officer}

\PlaceDateNote{St Bartholemew's hospital, London}{1992--1994 }{
Senior house officer in medicine}

\PlaceDateNote{Addenbrooke's hospital, Cambridge}{1992 }{
Senior house officer in neurosciences}

\PlaceDateNote{Royal London hospital}{1991 }{
House officer in medicine}

\PlaceDateNote{Princess Alexandra hospital, Harlow}{1990 }{
House officer in surgery}

\end{cvSection}

\begin{cvSection}{Research metrics}[
    \footnote{ From
    \url{https://scholar.google.com/citations?user=q12RP7AAAAAJ} as of July 7,
2016}]

{\bf Citations}: 10460 \\
{\bf h-index}: 30 \\
{\bf i10-index}: 37

\end{cvSection}

\begin{cvSection}{Neurology}

\begin{cvSubSection}{Neurology qualifications}

National training number in neurology (UK) \hfill {\em 1996} \\
Member of the royal college of physicians (UK) \hfill {\em 1994}

\end{cvSubSection}

\begin{cvSubSection}{Neurology teaching}

\PlaceDateNote{Oxford university}{1994--1995}{
Supervision of medical students in neuroanatomy}

\end{cvSubSection}

\begin{cvSubSection}{Neurology articles}

\printbibliography[heading=none,
    keyword=neurology,
notkeyword=omit]

\end{cvSubSection}

\end{cvSection}

\begin{cvSection}{Cognitive and motor neuroscience}

\begin{cvSubSection}{Neuroscience awards}

{\bf British brain and spine foundation training fellowship} \hfill {\em
1996--1999}

\end{cvSubSection}

\begin{cvSubSection}{Neuroscience research supervision}

\PlaceDateNote{Cambridge university BA final year projects}{2007--2008}
{Final year undergraduate projects in experimental psychology by Sam Burnand
and Rich Armstrong.  Projects on functional MRI of response selection.  Both
projects graded as first class}

\PlaceDateNote{Cambridge university PhD}{2001--2004}
{Jessica Grahn: {\em The functional anatomy of musical beat perception}.
Jessica is an associate professor in the Brain and Mind Institute, Western
University, Ontario}

\PlaceDateNote{Cambridge university PhD}{2000--2004}
{Katja Osswald: {\em The role of SMA and basal ganglia in motor learning,
mechanisms of apraxia and methods of functional MRI analysis}. Katja is an
associate lecturer at the department of psychology in York and an NHS clinical
psychologist}

\end{cvSubSection}

\begin{cvSubSection}{Neuroscience teaching}

\PlaceDateNote{Cambridge university}{2007--2008}{
Supervision in undergraduate neuroscience for Jesus college}

\end{cvSubSection}

\begin{cvSubSection}{Neuroscience articles}

\printbibliography[heading=none,
    keyword=movethink,
    keyword=article,
notkeyword=omit]

\end{cvSubSection}

\begin{cvSubSection}{Neuroscience abstracts}

\printbibliography[heading=none,
    keyword=movethink,
    keyword=abstract,
notkeyword=omit]

\end{cvSubSection}

\end{cvSection}

\begin{cvSection}{Imaging methods and statistics}

\begin{cvSubSection}{Methods research supervision}

\PlaceDateNote{Cambridge university post-doctoral research}{2002--2006}
{Ferath Kherif, working on multivariate statistics for clustering and
diagnostics of functional imaging data. Ferath is a principal investigator at
the Laboratory of Research in Neuroimaging, Lausanne, Switzerland}

\PlaceDateNote{Cambridge university post-doctoral research}{2001--2002}
{Alexandre Andrade, on brain surface-based functional MRI statistics, coherence
analysis.  Alexandre is assistant professor at the Physics Department of the
Faculty of Sciences of the University of Lisbon}

\end{cvSubSection}

\begin{cvSubSection}{Methods teaching}

\PlaceDateNote{\UCB}{2013--present}
{Organizer for ``practical neuroimaging'' post-graduate course -- combines
teaching of concepts behind imaging analysis with training in scientific
computing}

\PlaceDateNote{\UCB}{2015}
{Co-taught with Jarrod Millman.  Undergraduate and masters statistics course
on ``Reproducible and collaborative statistical data science''}

\PlaceDateNote{\UCB}{2008--present}
{Lecturer on functional MRI spatial processing and statistics for
post-graduate ``Functional MRI methodology seminar''}

\PlaceDateNote{\UCB}{2008--2011}
{Speaker for post-graduate ``Neuroimaging seminar series''}

\PlaceDateNote{Stanford University}{2013}
{Lecturer at the ``fMRI data analysis workshop''}

\PlaceDateNote{San Francisco}{2009}
{Lecturer on FMRIB Software Library course, speaking on ``Experimental
design''}

\PlaceDateNote{\CBU}{1999--2003, 2005--2008}
{Regular speaker at ``Imaging interest group'' seminar series on imaging
methods}

\PlaceDateNote{\CBU}{2007}
{Lecturer for short course on SPM}

\PlaceDateNote{Human brain mapping conference}{2004, 2006, 2007}
{Lecturer for introductory training course on functional MRI}

\PlaceDateNote{Oslo, Norway}{2005}
{Co-organized and co-taught with Ansgar Furst. 4-day course on FMRI analysis
using SPM software}

\PlaceDateNote{Yale}{2005}
{Faculty for course on anatomical and functional MRI analysis using SPM}

\PlaceDateNote{Paris, France}{2000--2003}
{Lecturer on statistics and spatial processing for functional imaging analysis
course}

\PlaceDateNote{Melbourne, Australia}{2001}
{Only teacher of 9 hours of lectures and 5 hours of practical sessions on
functional MRI analysis using SPM software}

\end{cvSubSection}

\begin{cvSubSection}{Selected methods tutorials}

\printbibliography[heading=none,
    filter=imagingOrStatistics,
    keyword=online,
notkeyword=omit]

\end{cvSubSection}

\begin{cvSubSection}{Methods articles}

\printbibliography[heading=none,
    keyword=methods,
    keyword=article,
notkeyword=omit]

\end{cvSubSection}

\begin{cvSubSection}{Methods abstracts}

\printbibliography[heading=none,
    keyword=methods,
    keyword=abstract,
notkeyword=omit]

\end{cvSubSection}

\end{cvSection}

\begin{cvSection}{Scientific computing}

\begin{cvSubSection}{Computing leadership}

\PlaceDateNote{Neuroimaging in Python project}{2004--present}
{Co-founder (with Jarrod Millman) of the neuroimaging in Python project (NIPY)
    \url{http://nipy.org}.  Co-author (with Jarrod Millman) of R01 grant to
    fund NIPY development (see below).  Set
    NIPY community development standards including BSD code license, version
    control, continuous integration testing on all major platforms, automated
    reporting of code test coverage, formal code review.  The NIPY
    organization \url{https://github.com/nipy} is now home to 12 neuroimaging
code projects.  Lead author and maintainer of \Pkg{nibabel} and \Pkg{nipy}
software projects; third contributor by code commits of \Pkg{dipy} software
project (see below).  99th centile personal ranking for scientific code impact
by Depsy.org \footnote{See: \url{http://depsy.org} and {\em Nature} 2016: 529,
115–116\label{depsy}}. I believe I am the author of the first fully
reproducible publication in neuroimaging (Aston, Brett and Turkheimer, 2006).
}

\PlaceDateNote{Scientific Python developer}{2004--present}
{Contributor to all the main scientific Python packages, including \Pkg{numpy,
    scipy, matplotlib, Cython, statsmodels}; responsible for maintenance and
    release of installers for many scientific Python packages, including
    \Pkg{numpy, scipy, matplotlib, Cython, statsmodels}; organization member
    of projects \Pkg{numpy, scipy, matplotlib, scikit-image, Python-pillow,
    MacPython} and the Python packaging
authority.}

\end{cvSubSection}

\begin{cvSubSection}{Computing grants}

\PlaceDateNote{NIH RO1 grant}{2007--2010}
{Co-author (with Jarrod Millman) of NIH grant 5R01MH081909-02 ``Continued
development and maintenance of the Neuroimaging In Python project''}

\end{cvSubSection}

\begin{cvSubSection}{Selected scientific software}

See \url{https://www.openhub.net/accounts/matthew-brett}

\PlaceDateNote{Dipy}{2009--present}
{Third-ranked developer by code commits.  Python package for analysis of
    diffusion MRI data.  69 contributors, 43,806 lines of code, estimated 11
    years of developer effort
    \footnote{\url{https://www.openhub.net/p/dipy}}. 99th centile for research
impact among all scientific R and Python projects \footref{depsy}.}

\PlaceDateNote{Nipy}{2006--present}
{Lead developer and maintainer.  Spatial processing and statistical analysis
of functional MRI data.  59 contributors, 69,522 lines of code, estimated 18
years of developer effort \footnote{\url{https://www.openhub.net/p/nipy}}.
97th centile for research impact among all R, Python projects
\footref{depsy}.}

\PlaceDateNote{Nibabel}{2005--present}
{Lead developer and maintainer. Reads and writes standard neuroimaging file
    formats.  46 contributors, 28,473 lines of code, estimated 7 years of
    developer effort \footnote{\url{https://www.openhub.net/p/nibabel}}.
    100th centile for research impact among all R, Python projects
\footref{depsy}.}

\PlaceDateNote{MarsBaR}{2003--present}
{Lead developer and maintainer. Region of interest analysis for functional
    imaging data.  3 contributors, representing 22,166 lines of code,
estimated 6 years of developer effort
\footnote{\url{https://www.openhub.net/p/marsbar}}.  The abstract for MarsBaR
has been cited 2127 times as of July 2016.}

\PlaceDateNote{Phiwave}{2004--2005}
{Lead developer and maintainer.  Wavelet analysis for spatial inference on
functional imaging data.  2 contributors, 5,367 lines of code, estimated
estimated 2 years of effort
\footnote{\url{https://www.openhub.net/p/phiwave}}.}

\end{cvSubSection}

\begin{cvSubSection}{Computing teaching}

See also: teaching on imaging methods and statistics.

\PlaceDateNote{\UCB}{2016}
{Certified as instructor for Software Carpentry workshops \footref{swc}}

\PlaceDateNote{Havana, Cuba}{2013}
{Invited speaker and teacher at the Latin-American summer school on
Neuroinformatics, speaking on ``The need and methods for reproducible
science''}

\PlaceDateNote{Lawrence Berkeley National Laboratory}{2012}
{Instructor for Software Carpentry
    \footnote{
        Software carpentry (\url{http://software-carpentry.org}) is an
        international project to teach scientists effective use of computing
        tools
\label{swc}}
boot-camp}

\end{cvSubSection}

\begin{cvSubSection}{Selected computing tutorials}

\printbibliography[heading=none,
    keyword=computing,
    keyword=online,
notkeyword=omit]

\end{cvSubSection}

\begin{cvSubSection}{Computing articles}

\printbibliography[heading=none,
    keyword=computing,
    keyword=article,
notkeyword=omit]

\end{cvSubSection}

\begin{cvSubSection}{Computing abstracts}

\printbibliography[heading=none,
    keyword=computing,
    keyword=abstract,
notkeyword=omit]

\end{cvSubSection}

\end{cvSection}

\begin{cvSection}{Service}

    {\em Ad-hoc reviewer for} NeuroImage; Human Brain Mapping; Journal of
    Cognitive Neuroscience; Neuroscience Letters; Clinical Neurophysiology;
    Journal of Neuroimaging; the Journal of Clinical and Experimental
    Neuropsychology; Frontiers in Neuroinformatics; Computing in Science and
    Engineering; Frontiers in Brain Imaging Methods; Frontiers in
    Neuroanatomy; Public Library of Science One.

\PlaceDateNote{\UCB}{2015--present}
{Organizer for the imaging analysis discussion group -- a weekly / bi-weekly
seminar series on analysis of imaging data}

\PlaceDateNote{\CBU}{2007--2008}
{Graduate committee member}

\PlaceDateNote{\CBU}{2001--2003, 2005--2008}
{Member of the imaging management committee}

\PlaceDateNote{\CBU}{2006--2008}
{Organizer of imaging methods journal club}

\PlaceDateNote{\CBU}{1999--2003}
{Organizer for the ``Imaging interest group'' -- a Cambridge-wide weekly
seminar on functional imaging analysis and results}

\end{cvSection}
\end{document}
