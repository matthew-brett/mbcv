% The system here for producing my publication list in sections uses biblatex
% and biber to customize the bibliography output. Sections defined by a
% combination of keyword= and custom filtering. See
% http://www.ctan.org/pkg/biblatex

\documentclass[11pt]{cv}
% Document margins
\usepackage[left=0.5in,top=0.5in,right=0.5in,bottom=0.5in]{geometry}
\usepackage[utf8]{inputenc}
\usepackage[T1]{fontenc}
\usepackage{mathpazo}
\usepackage{hyperref}
\usepackage{url}
\usepackage{footmisc}

\pagestyle{plain}
\pagenumbering{gobble}

\usepackage[defernumbers=true,
    bibstyle=authoryear,
    backend=biber,
    maxbibnames=100,
    url=false,
    sorting=ydnt,
    style=apa]{biblatex}

\usepackage{hyperref}
\hypersetup{ % turn off the boxes round URLs
    colorlinks,
    citecolor=black,
    filecolor=black,
    linkcolor=black,
    urlcolor=black
}

\addbibresource{matthew_brett.bib}
\addbibresource{cv.bib}

\AtEveryBibitem{\clearlist{language}} % clears language

% This stuff below was the result of trial and error to get the bibliography
% output to a) preserve the subsequent paragraph indentation as indicated by
% the adjustwidth environments, and b) To have the correct indentation for
% each entry in the bibliography.  The source from which I was working was
% http://tex.stackexchange.com/questions/46298/printing-bibliography-with-biblatex-in-tufte-handout-fullwidth-environment
\defbibenvironment{bibliography}
  {\list{}{%
          \leftmargin\bibhang
  }}
  {\endlist}
    {\item}

\AtEveryBibitem{\hskip-\bibhang}

% Select methods and statistics entries in bib database
\defbibfilter{methodsOrStatistics}{%
    keyword=methods or keyword=statistics
}

\newcommand{\PlaceDate}[2]{{\bf #1} \hfill {\em #2} \\}
\newcommand{\PlaceDateNote}[3]{{\bf #1} \hfill {\em #2} \\#3}
\newcommand{\LIS}{London Interdisciplinary School, UK}
\newcommand{\UoB}{University of Birmingham, UK}
\newcommand{\UCB}{University of California, Berkeley}
\newcommand{\CBU}{MRC Cognition and Brain Sciences Unit, Cambridge, UK}
\newcommand{\Pkg}[1]{{\tt #1}}

\begin{document}

{\huge \bf Matthew Brett}

London Interdisciplinary School \\
matthew.brett@gmail.com

\begin{cvSection}{Recent work}

\PlaceDateNote{\LIS}{2022--present }{
    Associate professor in data science}

\PlaceDateNote{\UoB}{2017--2021 }{
    Lecturer in data science}

\PlaceDateNote{\UCB}{2008--2017 }{
    Associate researcher, Brain Imaging Center}

\PlaceDateNote{\CBU}{2005--2008}{
    Senior investigator scientist}

\PlaceDateNote{\UCB}{2003--2005 }{
    Associate specialist in psychology}

\PlaceDateNote{\CBU}{1999--2003 }{
    Research associate in psychology}

\PlaceDateNote{
MRC Cyclotron Unit, Hammersmith Hospital / Physiology Laboratory, Oxford}
{1996--1999}
    {Research registrar in neurology}

\end{cvSection}

\begin{cvSection}{Education and qualifications}

{\bf Membership of the Royal College of Physicians} \hfill {\em 1994}

\PlaceDate{Royal London Hospital}{1987--1990 }
Bachelor of medicine and surgery

\PlaceDateNote{Cambridge University}{1984--1987 }{
BA Experimental psychology}

\end{cvSection}

\begin{cvSection}{Research metrics}

{\bf Citations}: 34728 \\
{\bf h-index}: 41 \\
{\bf i10-index}: 50

\end{cvSection}

\begin{cvSection}{Selected teaching}

\PlaceDateNote{Computing for data}{2019-20}
    {Two terms at \UoB; masters and neuroscience undergraduates; Python course
    based on Berkeley's {\it Foundations of Data Science}; all class materials
    published with open licence at
    \url{https://github.io/matthew-brett/cfd2019}.}

\PlaceDateNote{Functional MRI methods}{2016}
    {\UCB Neuroscience graduate students, teaching reproducible analysis of
    functional MRI analysis through coding in Python, Git version control and
    Github working process. Emphasis on Nibabel, Numpy, Scipy, Matplotlib. All
    materials online with open license at
    \url{https://bic-berkeley.github.io/psych-214-fall-2016/topics.html}.}

\PlaceDateNote{Reproducible and collaborative statistical data science}{2015}
    {\UCB; undergraduates and masters students; statistics / neuroimaging
    course taught with Python, Git and Github; emphasis on Numpy, Scipy,
    Matplotlib and Nibabel. Co-taught with Jarrod Millman for the statistics
    department.  We describe and assess the course in
    \cite{millman2018rcsds}.}

\end{cvSection}

\begin{cvSection}{Selected scientific computing}

\PlaceDateNote{Scientific Python: developer}{2004--present}
{Core contributor to Scipy; contributor to all main scientific Python
    packages, including \Pkg{numpy, scipy, matplotlib, Cython, statsmodels};
    organization member of projects \Pkg{numpy, scipy, matplotlib,
    scikit-image, Python-pillow, MacPython} and the Python packaging
    authority.}

\PlaceDateNote{Neuroimaging in Python project}{2004--present}
{Co-founder (with Jarrod Millman) of the neuroimaging in Python project (NIPY)
    \url{https://nipy.org}.  Nipy is now home to 12 neuroimaging
    code projects.}

\PlaceDateNote{\Pkg{nibabel}: lead author}{2008--present}
    {A foundation library for neuroimaging data formats; 1794 commits.}

\PlaceDateNote{\Pkg{NiPy}: lead author and maintainer}{2008--present} {Project
    implementing spatial processing and statistics for functional MRI
    data; 1657 commits.}

\PlaceDateNote{\Pkg{DiPy}: developer}{2009--present}
    {\Pkg{dipy} is a Python library for
    analysis of diffusion imaging data; 564 commits.}

\PlaceDateNote{\Pkg{MarsBaR}: lead author and maintainer}{2003--present}
    {Widely used region of interest analysis toolbox for functional imaging
    data in Matlab.  MarsBaR abstract cited 2796 times as of August 2020. 713
    commits.}

\end{cvSection}

\begin{cvSection}{Selected publications}

\nocite{harris2020array,
    virtanen2020scipy,
    millman2018rcsds,
    garyfallidis2014dipy,
    Poline2012,
    Poldrack2008,
    Millman2007,
    Brett2007,
    Saxe2006,
    Brett2002}

\printbibliography[heading=none]

\end{cvSection}

\end{document}
