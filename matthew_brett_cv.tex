% The system here for producing my publication list in sections uses biblatex
% and biber to customize the bibliography output. Sections defined by a
% combination of keyword= and custom filtering. See
% http://www.ctan.org/pkg/biblatex

\documentclass{cv}
% Document margins
\usepackage[left=0.75in,top=0.6in,right=0.75in,bottom=0.6in]{geometry}
\usepackage[utf8]{inputenc}
\usepackage[T1]{fontenc}
\usepackage{mathpazo}
\usepackage{hyperref}
\usepackage{url}
\usepackage{footmisc}

\pagestyle{plain}
\pagenumbering{arabic}

\usepackage[defernumbers=true,
    bibstyle=authoryear,
    backend=biber,
    maxbibnames=10,
    url=false,
    sorting=ydnt,
    style=apa]{biblatex}

\usepackage{hyperref}
\hypersetup{ % turn off the boxes round URLs
    colorlinks,
    citecolor=black,
    filecolor=black,
    linkcolor=black,
    urlcolor=black
}

\addbibresource{matthew_brett.bib}
\addbibresource{cv.bib}

\AtEveryBibitem{\clearlist{language}} % clears language

% This stuff below was the result of trial and error to get the bibliography
% output to a) preserve the subsequent paragraph indentation as indicated by
% the adjustwidth environments, and b) To have the correct indentation for
% each entry in the bibliography.  The source from which I was working was
% http://tex.stackexchange.com/questions/46298/printing-bibliography-with-biblatex-in-tufte-handout-fullwidth-environment
\defbibenvironment{bibliography}
  {\list{}{%
          \leftmargin\bibhang
  }}
  {\endlist}
    {\item}

\AtEveryBibitem{\hskip-\bibhang}

% Select methods and statistics entries in bib database
\defbibfilter{methodsOrStatistics}{%
    keyword=methods or keyword=statistics
}

\newcommand{\PlaceDate}[2]{{\bf #1} \hfill {\em #2} \\}
\newcommand{\PlaceDateNote}[3]{{\bf #1} \hfill {\em #2} \\#3}
\newcommand{\UCB}{University of California, Berkeley}
\newcommand{\UoB}{University of Birmingham\,}
\newcommand{\CBU}{MRC Cognition and Brain Sciences Unit, Cambridge}
\newcommand{\Pkg}[1]{{\tt #1}}

\begin{document}

\nocite{*}

{\huge \bf Matthew Brett}

School of Psychology \\
Birmingham University \\
m.brett@bham.ac.uk

\begin{cvSection}{Education}

\PlaceDate{Royal London Hospital}{1987--1990 }
Bachelor of medicine and surgery

\PlaceDateNote{Cambridge University}{1984--1987 }{
BA Experimental psychology}

\end{cvSection}

\begin{cvSection}{Work history}

\PlaceDateNote{\UoB}{2017--present }{
Lecturer in psychology and data science}

\PlaceDateNote{\UCB}{2008--2017 }{
Associate researcher at the Brain Imaging Center}

\PlaceDateNote{\CBU}{2005--2008}{
Senior investigator scientist}

\PlaceDateNote{\UCB}{2003--2005 }{
    Associate specialist in psychology}

\PlaceDateNote{\CBU}{1999--2003 }{
    Research associate in psychology}

\PlaceDateNote{
MRC Cyclotron Unit, Hammersmith Hospital / Physiology Laboratory, Oxford}
{1996--1999}
    {Research registrar in neurology}

\PlaceDateNote{Radcliffe Infirmary, Oxford}
{1995--1999}
{Neurology registrar}

\PlaceDateNote{National Hospital for Neurology, London}{1994--1995 }{
Neurology senior house officer}

\PlaceDateNote{St Bartholemew's Hospital, London}{1992--1994 }{
Senior house officer in medicine}

\PlaceDateNote{Addenbrooke's Hospital, Cambridge}{1992 }{
Senior house officer in neurosciences}

\PlaceDateNote{Royal London Hospital}{1991 }{
House officer in medicine}

\PlaceDateNote{Princess Alexandra Hospital, Harlow}{1990 }{
House officer in surgery}

\end{cvSection}

\begin{cvSection}{Awards and qualifications}

\PlaceDateNote{Turing Fellow}{August 2018--present}
    {Fellow of the Alan Turing Institute}

\PlaceDateNote{NIH RO1 grant}{2007--2010}
{Co-author (with Jarrod Millman) of NIH grant 5R01MH081909-02 ``Continued
development and maintenance of the Neuroimaging In Python project''}

{\bf British brain and spine foundation training fellowship} \hfill {\em
1996--1999}

{\bf Membership of the Royal College of Physicians} \hfill {\em 1994}

{\bf Open entrance scholarship to Cambridge University} \hfill {\em 1984}

\end{cvSection}

\begin{cvSection}{Data science teaching}

\PlaceDateNote{Resampling statistics textbook}{present}
    {Finalizing a contract with SAGE publishing for a new edition of a classic
    text ``Resampling: the new statistics'' by Julian Simon:
    \url{http://www.resample.com/intro-text-online}. The new edition will use
    R and Python programming languages to illustrate and implement the
    algorithms in the text}.

\PlaceDateNote{R teaching for GGM108: Research methods, field work, and data
    analysis}{2019--present}
    {Four weeks geography undergraduate teaching on R and data analysis; 250
    students, 8 hours of lectures, 10 hours of practicals}.

\PlaceDateNote{MSc teaching on atmospheric data processing}{2019--present}
    {Four weeks teaching on R and data analysis for MSc programmes on {\em
    Atmospheric Data Processing and Statistics} and {\em Air Quality Data
    Analysis and Interpretation}. 34 students, 8 hours of lectures /
    practicals.}

\PlaceDateNote{Data science for everyone WHM}{2018--present}
    {20 credit widening horizons module. 38 students learning data
    analysis and statistical inference using the Python programming language.
    Interactive textbook at \url{https://matthew-brett.github.io/dsfe}.}

\PlaceDateNote{Interactive beginners tutorial for R}{2018--present}
    {Full interactive beginners tutorial for the R programming language using
    the DataCamp platform, currently hosted at
    \url{https://www.datacamp.com/courses/r-from-scratch}. In collaboration
    with the Alan Turing Institute (ATI), refactored course so it can be deployed on
    \UoB and ATI servers.}

\PlaceDateNote{R teaching for BIO240: Communication and Skills in
    Biosciences}{2017--present}
    {250 students; two 5 hour assignments of self-paced study and assessment
    using R.}

\PlaceDateNote{Pilot courses for data science teaching}{September 2017--present}
    {Three pilot courses for teaching programming for data analysis to
    undergraduates with no previous experience of programming, the latter two
    funded by a \UoB Educational Enhancement Fund grant ``Strategies to engage
    students in programming for data analysis''.}

\PlaceDateNote{Peer learning for academic computing}{2017--preset}
    {Founder and co-organizer of ``The Hacker Within'', a monthly
    peer-learning group on academic computing.}

\PlaceDateNote{Reproducible and collaborative statistical data science}{2015}
    {40 undergraduates and 10 masters students, co-taught with Jarrod Millman
    for the statistics department at the \UCB.  We describe and assess the course
    in \cite{millman2018rcsds}.}

\PlaceDateNote{Data science methods for teaching imaging
    analysis}{1998--present}
    {I have been developing data science teaching methods since the start of
    my academic career, in 1996, beginning with online tutorials in Matlab
    (e.g.
    \url{http://imaging.mrc-cbu.cam.ac.uk/imaging/PrinciplesStatistics}), and
    developing into full courses implementing data science teaching methods in
    brain imaging, from 2013 (see below).}

\end{cvSection}

\begin{cvSection}{Other teaching}

\PlaceDateNote{\UCB}{2016}
    {Certified as instructor for Software Carpentry workshops}

\PlaceDateNote{\UCB}{2016}
{Lead instructor for Berkeley post-graduate course ``PSYCH 214 -- functional
MRI methods'' \url{https://bic-berkeley.github.io/psych-214-fall-2016}}

\PlaceDateNote{\UCB}{2013--2017}
{Organizer, designer and main teacher for ``practical neuroimaging''
post-graduate course -- combines teaching of concepts behind imaging analysis
with training in scientific computing}

\PlaceDateNote{\UCB}{2015--2017}
{Organizer: imaging analysis discussion group}

\PlaceDateNote{Havana, Cuba}{2013}
{Invited speaker and teacher at the Latin-American summer school on
Neuroinformatics, speaking on ``The need and methods for reproducible
science''}

\PlaceDateNote{\UCB}{2008--2017}
{Lecturer on functional MRI spatial processing and statistics for
post-graduate ``Functional MRI methodology seminar''}

\PlaceDateNote{Stanford University}{2013}
{Lecturer at the ``fMRI data analysis workshop''}

\PlaceDateNote{\UCB}{2008--2011}
{Speaker for post-graduate ``Neuroimaging seminar series''}

\PlaceDateNote{San Francisco}{2009}
{Lecturer on FMRIB Software Library course, speaking on ``Experimental
design''}

\PlaceDateNote{\CBU}{1999--2003, 2005--2008}
{Organizer (1999--2003) and regular speaker at ``Imaging interest group''
seminar series}

\PlaceDateNote{\CBU}{2007}
{Lecturer for short course on SPM}

\PlaceDateNote{Human Brain Mapping conference}{2004, 2006, 2007}
{Lecturer for introductory training course on functional MRI}

\PlaceDateNote{Oslo, Norway}{2005}
{Co-organized and co-taught with Ansgar Furst. 4-day course on FMRI analysis
using SPM software}

\PlaceDateNote{Yale}{2005}
{Faculty for course on anatomical and functional MRI analysis using SPM}

\PlaceDateNote{Paris, France}{2000--2003}
{Lecturer on statistics and spatial processing for functional imaging analysis
course}

\PlaceDateNote{Melbourne, Australia}{2001}
{Only teacher of 9 hours of lectures and 5 hours of practical sessions on
functional MRI analysis using SPM software}

\end{cvSection}

\begin{cvSection}{Scientific computing}

\PlaceDateNote{Scientific Python: developer}{2004--present}
{Contributor to all main scientific Python packages, including
    \Pkg{numpy, scipy, matplotlib, Cython, statsmodels}; organization member
    of projects \Pkg{numpy, scipy, matplotlib, scikit-image, Python-pillow,
MacPython} and the Python packaging authority.}

\PlaceDateNote{Neuroimaging in Python project}{2004--present}
{Co-founder (with Jarrod Millman) of the neuroimaging in Python project (NIPY)
    \url{http://nipy.org}.  Nipy is now home to 12 neuroimaging
    code projects}

\PlaceDateNote{\Pkg{nibabel}: lead author}{2008--present}
    {A foundation library for neuroimaging data formats}

\PlaceDateNote{\Pkg{NiPy}: lead author and maintainer}{2008--present}
    {Project implementing spatial processing and statistics for functional MRI
    data.}

\PlaceDateNote{\Pkg{DiPy}: developer}{2009--present}
    {Third contributor by commits.  \Pkg{dipy} is a Python library for
    analysis of diffusion imaging data}.

\PlaceDateNote{\Pkg{MarsBaR}: lead author and maintainer}{2003--present}
{Widely used region of interest analysis toolbox for functional imaging data
    in Matlab.  MarsBaR abstract has been cited 2475 times as of March 2019.}

    \PlaceDateNote{\Pkg{Phiwave}: lead author and maintainer}{2004--2005}
{Matlab toolbox implementing wavelet analysis for spatial inference on
    functional imaging data.}

\end{cvSection}

\begin{cvSection}{Research supervision}

\PlaceDateNote{\CBU}{2007--2008}
{Member of the Graduate Committee}

\PlaceDateNote{Cambridge University BA}{2007--2008}
{Final year undergraduate projects in experimental psychology}

\PlaceDateNote{Cambridge University post-doctoral research}{2002--2006}
{Ferath Kherif, working on multivariate statistics for clustering and
diagnostics of functional imaging data. Ferath is a principal investigator at
the Laboratory of Research in Neuroimaging, Lausanne, Switzerland}

\PlaceDateNote{Cambridge University PhD}{2001--2004}
{Jessica Grahn: {\em The functional anatomy of musical beat perception}.
Jessica is an associate professor in the Brain and Mind Institute, Western
University, Ontario}

\PlaceDateNote{Cambridge University PhD}{2000--2004}
{Katja Osswald: {\em The role of SMA and basal ganglia in motor learning,
mechanisms of apraxia and methods of functional MRI analysis}. Katja is an
associate lecturer at the department of psychology in York and an NHS clinical
psychologist}

\PlaceDateNote{Cambridge University post-doctoral research}{2001--2002}
{Alexandre Andrade, on brain surface-based functional MRI statistics, coherence
analysis.  Alexandre is assistant professor at the Physics Department of the
Faculty of Sciences of the University of Lisbon}

\end{cvSection}

\begin{cvSection}{Reviewer}

    PLOS One, F1000, NeuroImage; Human Brain Mapping; Journal of Cognitive
    Neuroscience; Neuroscience Letters; Clinical Neurophysiology; Journal of
    Neuroimaging; the Journal of Clinical and Experimental Neuropsychology;
    Frontiers in Neuroinformatics; Computing in Science and Engineering;
    Frontiers in Brain Imaging Methods; Frontiers in Neuroanatomy.

\end{cvSection}

\begin{cvSection}{Research metrics}[
    \footnote{ From
    \url{https://scholar.google.com/citations?user=q12RP7AAAAAJ} as of March
    19, 2019}]

{\bf Citations}: 14000 \\
{\bf h-index}: 33 \\
{\bf i10-index}: 39

\end{cvSection}

\begin{cvSection}{Articles}

\printbibliography[heading=none,
    keyword=article,
    notkeyword=omit]

\end{cvSection}

\begin{cvSection}{Selected abstracts}

\printbibliography[heading=none,
    keyword=abstract,
    notkeyword=omit]

\end{cvSection}

\end{document}
